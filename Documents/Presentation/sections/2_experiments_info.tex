
\begin{frame}{Literatura}
  Do zrozumienia problemu wykorzystano następujące pozycje:
    \begin{enumerate}
      \item \url{https://arxiv.org/pdf/1911.08265}
      \item \url{https://www.youtube.com/watch?v=c8SLNEpFSrs}
      \item \url{https://github.com/jachoo/pg-beamer}
      \item \url{https://github.com/jachoo/pg-beamer}
    \end{enumerate}
\end{frame}
  \begin{frame}{Przydatne repozytoria}
    Istnieje wiele koncepcji testowania oprogramowania, jedną z nich jest testowanie na podstawie właściwości \cite{pbt_bib}.
    \onslide<2->{Dziś dowiecie się:
        \begin{enumerate}
        \item na czym polega testowanie na podstawie właściwości.
        \item Czym różni się ono od klasycznego podejścia do testowania.
        \item Jakie są strategie testowania na podstawie właściwości. 
        \item Jak przy wykorzystaniu konceptu QuickCheck znaleźć wartości początkowe nieprzechodzące testy.
        \end{enumerate}
    }
  \end{frame}
  \begin{frame}{Framework}
    Istnieje wiele koncepcji testowania oprogramowania, jedną z nich jest testowanie na podstawie właściwości \cite{pbt_bib}.
    \onslide<2->{Dziś dowiecie się:
        \begin{enumerate}
        \item na czym polega testowanie na podstawie właściwości.
        \item Czym różni się ono od klasycznego podejścia do testowania.
        \item Jakie są strategie testowania na podstawie właściwości. 
        \item Jak przy wykorzystaniu konceptu QuickCheck znaleźć wartości początkowe nieprzechodzące testy.
        \end{enumerate}
    }
  \end{frame}
  \begin{frame}{Dataset}
    Istnieje wiele koncepcji testowania oprogramowania, jedną z nich jest testowanie na podstawie właściwości \cite{pbt_bib}.
    \onslide<2->{Dziś dowiecie się:
        \begin{enumerate}
        \item na czym polega testowanie na podstawie właściwości.
        \item Czym różni się ono od klasycznego podejścia do testowania.
        \item Jakie są strategie testowania na podstawie właściwości. 
        \item Jak przy wykorzystaniu konceptu QuickCheck znaleźć wartości początkowe nieprzechodzące testy.
        \end{enumerate}
    }
  \end{frame}
  \begin{frame}{Co będzie przedmiotem eksperymentu}
    Istnieje wiele koncepcji testowania oprogramowania, jedną z nich jest testowanie na podstawie właściwości \cite{pbt_bib}.
    \onslide<2->{Dziś dowiecie się:
        \begin{enumerate}
        \item na czym polega testowanie na podstawie właściwości.
        \item Czym różni się ono od klasycznego podejścia do testowania.
        \item Jakie są strategie testowania na podstawie właściwości. 
        \item Jak przy wykorzystaniu konceptu QuickCheck znaleźć wartości początkowe nieprzechodzące testy.
        \end{enumerate}
    }
  \end{frame}
  \begin{frame}{Metryki i baseline}
    Istnieje wiele koncepcji testowania oprogramowania, jedną z nich jest testowanie na podstawie właściwości \cite{pbt_bib}.
    \onslide<2->{Dziś dowiecie się:
        \begin{enumerate}
        \item na czym polega testowanie na podstawie właściwości.
        \item Czym różni się ono od klasycznego podejścia do testowania.
        \item Jakie są strategie testowania na podstawie właściwości. 
        \item Jak przy wykorzystaniu konceptu QuickCheck znaleźć wartości początkowe nieprzechodzące testy.
        \end{enumerate}
    }
  \end{frame}
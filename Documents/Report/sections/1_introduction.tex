\section{Jak działa MuZero}
  MuZero działa w oparciu o zaawansowany model uczenia maszynowego, który łączy w sobie planowanie, uczenie się oraz symulację, aby optymalizować podejmowanie decyzji w różnych środowiskach. Kluczowe cechy działania MuZero to:
  \begin{itemize}
    \item  Modelowanie wewnętrzne - W przeciwieństwie do wcześniejszych algorytmów (np. AlphaGo czy AlphaZero), MuZero nie wymaga znajomości zasad gry ani pełnego modelu środowiska. Tworzy własną wewnętrzną reprezentację świata na podstawie danych wejściowych.
    \item Planowanie - MuZero wykorzystuje procesy podobne do wyszukiwania Monte Carlo Tree Search (MCTS), aby symulować różne scenariusze i przewidywać najbardziej korzystne działania. Wykorzystuje przy tym trzy modele:
    \begin{itemize}
      \item Model przewidujący wartość bieżącego stanu.
      \item Model przewidujący nagrodę za przejście w kolejny stan.
      \item Model przewidujący następny stan na podstawie wykonanej akcji.
    \end{itemize}

    \item Uczenie się z danych - System uczy się na podstawie danych historycznych oraz symulacji, aby ulepszać swoją strategię i lepiej przewidywać skutki akcji.
    \item Zastosowanie w różnych środowiskach - Może być stosowany w grach wideo, takich jak Atari, czy w bardziej złożonych środowiskach, takich jak Doom.
  \end{itemize}
  \addImage{images/MuZero.png}
  
\section{Podsumowanie}
    Mimo że sam pomysł muzero ma bardzo duży potencjał, badaczom nie udało się go w pełni odtworzyć. Większość problemów napotkanych w odtwarzaniu można przypisać niedoborom sprzętowym towarzyszącym podczas prób uczenia. 
    Widać jednak, że metoda ta ma zastosowanie, chociażby w cartpole, czy gridworld, które aż tak dużych wymagań co do danych nie miały. Najbardziej obiecującą okazały się gry crazy~climber i atlantis, gdzie agenci wykazywali się pewną inteligencją w ruchach.